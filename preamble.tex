%%%%%%%%%%%%%%%%%%%%%%%%%%%%%%%%%%%%%%%% 모든 패키지, 사용자 정의 매크로 호출 
% 문서 초기 설정
\usepackage{geometry} % 페이지 용지 설정

% 공통
\usepackage{xparse} % LaTeX3 언어 핵심 함수 제공 - 함수형 스타일, 내부 처리
\usepackage{expl3} % exp13 위에 구축된 고급 명령어 정의
\usepackage[hidelinks]{hyperref} % 참조나 목차에서 하이퍼링크가 가능하게 함
\usepackage{xifthen} % if 조건문 사용

% 폰트, 서식
\usepackage{luatexko} % 한글 폰트 사용. kotex의 lua 전용 후속 버전
\usepackage{fontspec} % 글꼴 강제 기울이기(한글 폰트 기울임)
\usepackage{enumitem} % 수동증가 번호 카운트
\usepackage[english]{babel} % 영어 하이픈 자동 삽입

% 색상
\usepackage{xcolor} % 색상
\usepackage{transparent} % lua만 됨. xcolor필요
\usepackage{luacolor,lua-ul} % 한글 형광펜 줄바꿈 등 완전 지원

% 이미지
\usepackage{graphicx} % 이미지 삽입
\DeclareGraphicsExtensions{.pdf, .png, .jpg} % 그림 확장자 생략
\usepackage{subcaption} % 이미지 묶어서 삽입하는 subfigure 사용

% 페이지
\usepackage{fancyhdr} % 머리말, 꼬리말
\setlength{\headheight}{18pt} % 헤더 높이 설정
\usepackage{xr-hyper} % 안쓰는 chapter에서 식 참조(aux 파일)할 때 사용
\usepackage{import} % main이 chapter 호출, cahpter가 section 호출(main에 대해 상대경로 사용)

% 캡션, 링크
\usepackage{aliascnt} % 고유 참조 이름(equation, figure 등을 커스텀)
\usepackage{caption} % 캡션 스타일 조정

%% 수식
\usepackage{calc} % 너비, 간격 등 계산
\usepackage{amsmath, amssymb, mathtools} % 본문에서 수학 식 사용
\usepackage{siunitx} % 단위 표기
\usepackage[f]{esvect} % 벡터 모양 확장(모양 a ~ h, 기본은 d)
% \vv{문자} 으로 쓰고, 첨자에 표시 안하고 싶으면 vv*{문자}{첨자}사용

% 표, 디자인
\usepackage[most]{tcolorbox} % 박스 만들기 - box.sty에서 필수
\usepackage{float} % 표/그림 위치 고정
\usepackage{booktabs} % 표 상단선 등 고급 표 기능 \toprule, \midrule, \bottomrule
\usepackage{longtable} % 페이지 넘어가는 긴 표
\usepackage{tabularx} % 표에서 너비 자동 조정
\usepackage{multirow} % 표에서 여러 행 병합
\usepackage{tikz} % 사각박스, 선긋기(교사용 메모 박스)
\usetikzlibrary{calc} % tikz에서 cal 사용
\usepackage{tikzpagenodes} % 사각박스, 선긋기(목차로 이동하는 박스)
\usepackage{array}        % 열 정렬 커스터마이징
\usepackage{colortbl}     % 셀 색칠 % 컴파일이 일어나는 main 기준으로 두 단계 올라가서 다시 내려옴
%%%%%%%%%%%%%%%%%%%%%%%%%%%%%%%%%%%%%%%% 보통 페이지

% 학습지용 양단 선 긋기. tikz 필요
\usetikzlibrary{backgrounds}
\newif\ifdrawlines
\drawlinesfalse
\AddToHook{shipout/background}{%
  \ifdrawlines
    \begin{tikzpicture}[remember picture, overlay]
      \ifodd\value{page}
        \draw[gray, line width=1pt] 
          ([xshift=-55mm,yshift=-20mm]current page.north east) -- 
          ([xshift=-55mm,yshift=20mm]current page.south east);
      \else
        \draw[gray, line width=1pt] 
          ([xshift=55mm,yshift=-20mm]current page.north west) -- 
          ([xshift=55mm,yshift=20mm]current page.south west);
      \fi\end{tikzpicture}\fi}

% 페이지 번호 제거 함수. nopagenumber
\newcommand{\nopagenumber}{
  \fancypagestyle{plain}{
    \fancyhf{}
    \renewcommand{\headrulewidth}{0pt}
    \renewcommand{\footrulewidth}{0pt}}
  \fancypagestyle{fancy}{
    \fancyhf{}
    \renewcommand{\headrulewidth}{0pt}
    \renewcommand{\footrulewidth}{0pt}}
  \pagestyle{fancy}}

% 페이지 번호 표시. pagenumber.
\newcommand{\pagenumber}{
  \fancypagestyle{plain}{ % plain이든 fancy든 모두 작동하도록 함
    \fancyhf{} % 초기화 먼저
    \fancyfoot[LE]{\hspace*{-5mm}\thepage} % 짝수는 왼쪽에서 더 밀기
    \fancyfoot[RO]{\thepage\hspace*{-5mm}} % 홀수는 오른쪽에서 더 밀기
    \renewcommand{\headrulewidth}{0pt}} % 상단 선 지우기
  \fancypagestyle{fancy}{ 
    \fancyhf{}
    \fancyfoot[LE]{\hspace*{-5mm}\thepage}
    \fancyfoot[RO]{\thepage\hspace*{-5mm}}
    \renewcommand{\headrulewidth}{0pt}}
  \pagestyle{fancy}}

% 여기서부터 페이지 번호를 1로 설정하는 명령
\newcommand{\resetpagenumber}{\setcounter{page}{1}}

% 페이지 하단의 꼬릿말 위치에 하이퍼링크(목차로 이동) 박스를 추가하는 명령. tikz, hyperref 필요
\newcommand{\linkbox}{
  \AddToHook{shipout/background}{%
    \begin{tikzpicture}[remember picture, overlay]
      \node[anchor=south west, inner sep=0pt] at ([yshift=15mm]current page.south west) {%
        \hyperlink{toc}{\phantom{\rule{\paperwidth}{5mm}}}};
    \end{tikzpicture}}}

% 여백 균등조절 자동으로 하지 않음.
% book의 경우 어느 정도 양이 차면 아래에 붙고 균등분배되는데 그것을 해제하는 것
% 다른 요소가 영향 미치고 간섭할 수 있음
\raggedbottom

% 페이지 애매하게 넘어갈 경우 addlines으로 한 두줄 강제로 확대함. 현재 페이지만 적용됨
\newcommand{\addlines}[1]{\enlargethispage{#1\baselineskip}}

%%%%%%%%%%%%%%%%%%%%%%%%%%%%%%%%%%%%%%%% chapter 형식 변경

%% chapter 타이틀 글자 너무 아래로 내리지 않게 살짝 올림
\makeatletter
% 번호 있는 챕터(\chapter{})의 제목 위치를 위로 올림
\renewcommand{\@makechapterhead}[1]{%
  \vspace*{-5pt} % 페이지 상단에서 챕터 제목까지 간격 줄이기
  {\parindent \z@            % 문단 들여쓰기 제거
   \raggedright \normalfont  % 왼쪽 정렬, 일반 글꼴
   \huge\bfseries \@chapapp\space \thechapter % 'Chapter 1' 등의 챕터 번호 출력
   \par\nobreak              % 문단 끝 + 줄 바꿈 금지
   \vskip 20pt               % 챕터 번호와 제목 사이 간격
   \Huge \bfseries #1\par\nobreak % 제목 본문
   \vskip 40pt}}               % 제목 아래 본문과의 간격
% 번호 없는 챕터(\chapter*{})의 제목 위치를 위로 올림 (예: contents, appendix 등)
\renewcommand{\@makeschapterhead}[1]{%
  \vspace*{-5pt} % 페이지 상단에서 챕터 제목까지 간격 줄이기
  {\parindent \z@            % 문단 들여쓰기 제거
   \raggedright \normalfont  % 왼쪽 정렬, 일반 글꼴
   \Huge\bfseries #1\par\nobreak % 제목 본문
   \vskip 40pt}}               % 제목 아래 본문과의 간격
\makeatother

%% 챕터 번호매김 - 0이면 chapter, 1이면 section, 2면 subsecion ,,,, 까지 번호매김
\setcounter{secnumdepth}{1} 

%% section 변경
\makeatletter
\renewcommand\section{\@startsection{section}{1}{\z@}
  {-3.25ex\@plus -1ex \@minus -.2ex} % 제목 위 여백
  {1.5ex \@plus .2ex} % 제목 아래 여백
  {\normalfont\fontsize{17pt}{17pt}\bfseries\selectfont}} % 글자 크기, 줄간격, 굵게
\makeatother

%% subsection 변경(* 붙이고*(선택) 글자 크기 변경)
\makeatletter
\renewcommand\subsection{\@startsection{subsection}{2}{\z@}
  {-3.25ex\@plus -1ex \@minus -.2ex}
  {1.5ex \@plus .2ex}
  {\normalfont\fontsize{15pt}{15pt}\bfseries\selectfont}}
%\let\oldsubsection\subsection % 기호 *을 앞에 붙이도록 함
%\renewcommand{\subsection}[1]{\oldsubsection{*\hspace{2pt}#1}}
\makeatother

%% subsubsection 변경(* 붙이고*(선택) 글자 크기 변경)
\makeatletter
\renewcommand\subsubsection{\@startsection{subsubsection}{3}{\z@}
  {-3.25ex\@plus -1ex \@minus -.2ex}
  {1.5ex \@plus .2ex}
  {\normalfont\fontsize{12pt}{12pt}\bfseries\selectfont}}
\let\oldsubsubsection\subsubsection % 기호 *을 앞에 붙이도록 함
\renewcommand{\subsubsection}[1]{\oldsubsubsection{*\hspace{2pt}#1}}
\makeatother

% 글 마감할 때 뒷표지 만들기
% #1 = 홀수페이지 내용, #2 = 짝수페이지 가운데 큰 글, #3 = 짝수페이지 하단 작은 글
\newcommand{\makecover}[2]{%
  \ifodd\value{page} % 홀수 끝인 경우 바로 커버 삽입
    \newpage
  \else % 짝수 끝인 경우 빈페이지 삽입
    \cleardoublepage
     \ % 빈 내용 삽입
  \fi
  \thispagestyle{empty} % 페이지 번호 없애기
  \drawlinesfalse % 선 긋기 중단
  \restoregeometry % 용지 설정 복원
  \vspace*{\fill}
  \begin{center}{\large #1}\end{center}
  \vspace*{\fill}
  \begin{center}{\small #2}\end{center}
  \thispagestyle{empty} % 페이지 번호 없애기
  \newpage}
%% 문단 정렬, 색상, 글자 스타일, 굵기, 기본 강조 명령어 등을 정의


%%%%%%%%%%%%%%%%%%%%%%%%%%%%%%%%%%%%%%%% 줄간격, 여백 설정
%% 줄 간격 설정 - 기본적으로 1.2배가 곱해짐 (10pt인 경우 1.2 x 10pt).
\linespread{1.1} % 추가로 더 넓힐 간격(1.2가 들어가면 1.44배가 됨.)

%% 기본 단락 여백 제거 - 문단 설정
\setlength{\parskip}{0pt} % 문단사이 간격 (기본값 0, 글자높이의 절반만큼 띄우려면 0.5em)
\setlength{\parindent}{0pt} % 문단 첫 들여쓰기 없음(글자 너비 절반인 0.5em 해도 됨)

%% 기본 줄간격 명령어 변경. - [\bs], [-0.5\bs]등으로 작성 가능
\newcommand{\bs}{\baselineskip}

%% 기본 줄바꿈 무효 + 보정값(대괄호) (기본은 0pt)
% 메모 삽입 등을 위하여 문단 마지막마다 \zz로 끝냄. \zz[2pt]쓰면 추가 2pt 간격 확보
% 수평 모드면 \\로, 수직 모드면 \vspace로 처리하게 함
% \zzz는 \\와 완전히 동일함. 즉, \zz[\bs]임. \zzz[5pt]는 추가 5pt 확장
\newcommand{\zz}[1][0pt]{\ifhmode \\[-\bs + #1] \else \vspace*{\dimexpr -\bs + #1\relax} \fi}
\newcommand{\zzz}[1][0pt]{%
  \ifvmode \vspace*{\bs + #1} % 수직 모드일 때는 빈 줄 간격
  \else \\[#1] \fi} % 그 외 본문(ifhmode), 수식(ifmmode), 표 내부 등 모든 모드에서 \\처럼 동작


%%%%%%%%%%%%%%%%%%%%%%%%%%%%%%%%%%%%%%%% 텍스트 스타일 - 볼드, 이탤릭
% 한글 폰트에 볼드체가 없는 경우가 있음. 기본적으로 bold체로 바꾸고, 한글은 FakeBold, 강도를 숫자로 조정
\newcommand{\bd}[1]{\ifmmode\bm{#1} % 수식 모드에서는 \bm으로 처리
  \else{\addhangulfontfeatures{AutoFakeBold=4}\textbf{#1}}\fi}
% 한글 폰트에 이탤릭체가 대부분 없음. 기본적으로 이탤릭체로 바꾸고, 한글은 FakeSlant, 강도를 숫자로 조정
\newcommand{\itl}[1]{{\addhangulfontfeatures{AutoFakeSlant=.3}\textit{#1}}}


%%%%%%%%%%%%%%%%%%%%%%%%%%%%%%%%%%%%%%%% 색상, 형광펜
%% 색상 정의 % xcolor, transparent 필요
\definecolor{myred}{RGB}{255,70,70}
\definecolor{myorange}{RGB}{255,170,90}
\definecolor{myyellow}{RGB}{255,230,40}
\definecolor{mylime}{RGB}{160,240,120}
\definecolor{mygreen}{RGB}{60,160,70}
\definecolor{myblue}{RGB}{70,90,230}
\definecolor{mygray}{RGB}{160,160,160}
\definecolor{mywhite}{RGB}{255,255,255}

%% 색상 텍스트
\newcommand{\red}[1]{\textcolor{myred}{#1}}
\newcommand{\orange}[1]{\textcolor{myorange}{#1}}
\newcommand{\yellow}[1]{\textcolor{myyellow}{#1}}
\newcommand{\lime}[1]{\textcolor{mylime}{#1}}
\newcommand{\green}[1]{\textcolor{mygreen}{#1}}
\newcommand{\blue}[1]{\textcolor{myblue}{#1}}
\newcommand{\gray}[1]{\textcolor{mygray}{#1}}
\newcommand{\white}[1]{\textcolor{mywhite}{#1}}

%% 형광펜, 강조 스타일 정의 - luacolor,lua-ul 쓰면 매우 강력
\newcommand{\hlr}[1]{\highLight[myred]{#1}}
\newcommand{\hlo}[1]{\highLight[myorange]{#1}}
\newcommand{\hly}[1]{\highLight[myyellow]{#1}}
\newcommand{\hll}[1]{\highLight[mylime]{#1}}
\newcommand{\hlgr}[1]{\highLight[mygreen]{#1}}
\newcommand{\hlb}[1]{\highLight[myblue]{#1}}
\newcommand{\hlgy}[1]{\highLight[mygray]{#1}}
\newcommand{\hlw}[1]{\highLight[mywhite]{#1}}


%%%%%%%%%%%%%%%%%%%%%%%%%%%%%%%%%%%%%%%% 자동 넘버링, an, bn, bul, arw
% biglabel 함수 지정. 볼드체 12pt
\newcommand{\biglabel}[1]{\bd{\fontsize{12pt}{12pt}\selectfont #1}}

% 화살표 지정. 본래 수학모드에서 쓰이지만 본문에서도 쓰이도록 함
\newcommand{\arrow}{\ensuremath{\,\,\,\longrightarrow\,\,\,}}

%% 수동 증가 번호 커맨드 제작. 함수명은 an(숫자점), bn(괄호숫자), cn(괄호영어)
% enumitem 패키지 필요, 항상 문단을 끊고(\par) 시작함. 마지막에는 \par로 닫아줘야 들여쓰기 종료됨.
% an의 경우 1em 박스에 숫자 우측(r)정렬하여 두자리가 되도 붙음. 간격은 \,로 삽입함. 들여쓰기 동일하게 1em 줌
% 들여쓰기를 #1로 받고, 맨 앞에서 1em, 1.6em 등으로 수정 - 사용할 시 변경 가능 (예) \an[2em] 내용입력

% anset, bnset, cnset으로 카운트 초기화
% anset{5}를 입력하면 5까지 이미 카운트되도록 함. cnset{d}를 입력하면 d까지 카운트하여 e부터 나오도록 함.

% an (숫자점)
\newcounter{anumcounter} % an
\NewDocumentCommand{\anset}{g}{\IfNoValueTF{#1}{\setcounter{anumcounter}{0}}{\setcounter{anumcounter}{#1}}}
\newcommand{\an}[1][1em]{\par\stepcounter{anumcounter}\makebox[#1][r]{\theanumcounter.\,}\hangindent=#1 \hangafter=1\relax}

% bn (괄호숫자)
\newcounter{bnumcounter} % bn
\NewDocumentCommand{\bnset}{g}{\IfNoValueTF{#1}{\setcounter{bnumcounter}{0}}{\setcounter{bnumcounter}{#1}}}
\newcommand{\bn}[1][1.6em]{\par\refstepcounter{bnumcounter}\makebox[#1][r]{(\arabic{bnumcounter})\,}\hangindent=#1 \hangafter=1\relax}

% cn (괄호영어)
\newcounter{cnumcounter} % cn
% cnset을 위해 먼저 숫자 영어 변환
\newcommand{\lettertonum}[1]{\number\numexpr`#1-`a+1\relax} % 알파벳(a..z)→숫자(1..26)
\newcommand{\numtoletter}[1]{\char\numexpr`a-1+#1\relax} % 숫자(1..26)→알파벳(a..z)
\NewDocumentCommand{\cnset}{g}{\setcounter{cnumcounter}{\IfNoValueTF{#1}{0}{\ifcat a#1 \number\numexpr`#1-`a+1\relax \else #1\fi}}}
\newcommand{\cn}[1][1.6em]{\par\refstepcounter{cnumcounter}\makebox[#1][r]{(\alph{cnumcounter})\,}\hangindent=#1 \hangafter=1\relax}

% dn (수동) - \dn{(가)} 이런식으로 쓰면 (가) 하고 들여쓰기 됨. 뒤에 \ignorespaces으로 공백 무시함(띄어쓰기 가능)
% \dn 으로만 쓰면 기본 여백으로 들여쓰기됨. \dn[2em] 등으로 간격만 줄 수도 있음
\NewDocumentCommand{\dn}{ O{1.6em} g }{\par\makebox[#1][r]{\IfNoValueTF{#2}{}{#2\,}}\hangindent=#1 \hangafter=1\relax\ignorespaces}


% bul (글머리 점) bullet은 커서 크기 줄임, bn과 들여쓰기 동일하게 함
\newcommand{\bul}[1][1.6em]{\par\makebox[#1][r]{\raisebox{0.25ex}{\scalebox{0.5}{$\bullet$}}\,\,}\hangindent=#1 \hangafter=1\relax}

% arw (화살표)
\newcommand{\arw}[1][3em]{\par\makebox[#1][r]{\arrow}\hangindent=#1 \hangafter=1\relax}






%% 여러 줄도 가능한 가운데 정렬 텍스트 매크로
\newcommand{\ctt}[1]{\par{\centering #1 \\}}

% 1로 자동조사 사용, 0으로 취소함. luatexko의 옵션
\josaignoreparens=1 % \은, \이, \으로 는, 가, 로 등으로 자동 수정됨

% 인라인 수식모드 $수식입력$ 안에서 디스플레이스타일 빠르게 지정
\newcommand{\dps}{\displaystyle}



% 불완전미분기호 dbar 생성
\newcommand{\dbar}{d\hspace*{-0.08em}\bar{}\hspace*{0.1em}}




%% 긴 URL, 저자명의 경우 줄바꿈
\PassOptionsToPackage{hyphens}{url}  % url 패키지에 옵션만 전달
\Urlmuskip=0mu plus 1mu

%% 영어 하이픈 자동 삽입 - \usepackage[english]{babel} 필요
\pretolerance=100           % 1차 시도: 띄어쓰기만으로 줄맞춤 (낮을수록 더 많이 줄바꿈)
\tolerance=500              % 2차 시도: 하이픈·늘림 포함 줄맞춤 (높을수록 더 많이 사용)
\emergencystretch=1em       % 최후: 여백 추가로 억지 줄맞춤
\hyphenpenalty=1000         % 자동 하이픈을 거의 안 씀 (값이 높을수록 하이픈을 덜 씀)
\exhyphenpenalty=50         % 수동 하이픈(\-)은 비교적 잘 씀
%% 박스, 틀, 테두리 등 시각적 요소 정의
%% 박스 스타일 정의 - 패키지 불러와야 newtcolorbox 쓸 수 있음

%%%%%%%%%%%%%%%%%%%%%%%%%%%%%%%%%%%%%%%% 공통 스타일 정의
% 박스 공통 스타일
\tcbset{boxbase/.style={
  boxrule=1pt, arc=1mm, % 테두리선 두께, 둥근 모서리 반경
  top=4pt, bottom=4pt, left=4pt, right=4pt, % 상하좌우 여백
  colback=white, % 추가 입력 안할 시 본문 배경은 흰색
  before upper={ % 박스 내부 여백 설정
  \setlength{\lineskiplimit}{2pt} % 인라인 위아래 간격이 임계값보다 작으면
  \setlength{\lineskip}{5pt} % 추가 간격 확보(수식 입력 등에 유용함)
  \setlength{\abovedisplayskip}{5pt} % 일반적인 디스플레이 수식 여백 설정
  \setlength{\belowdisplayskip}{5pt}
  \setlength{\abovedisplayshortskip}{5pt} % 특수상황(수식 앞줄이 짧게 끝나는 경우)
  \setlength{\belowdisplayshortskip}{5pt}}}} % 특수상황(수식 아랫줄이 짧게 끝나는 경우)

% 박스 제목 있는 경우 위아래 여백 미리 확보 - 수직이동량, 위쪽여백, 아래쪽여백, 내용으로 선언
\newcommand{\boxtitle}[1]{\raisebox{0pt}[9pt][2pt]{#1}}


%%%%%%%%%%%%%%%%%%%%%%%%%%%%%%%%%%%%%%%% 기본 단순박스 제작
% 회색 박스 (설명, 정리 등)
\newtcolorbox{graybox}{boxbase, % 공통 서식 먼저 호출
  colframe=gray!50}   % 테두리(기본 제목 배경) - 회색(50%)
% 강조용 빨간 테두리 박스
\newtcolorbox{redbox}{boxbase,
  colframe=red}       % 테두리(기본 제목 배경) - 빨간색
% 강조용 파란 배경 박스
\newtcolorbox{bluebox}{boxbase,
  colback=blue!5,     % 본문 배경 - 파란색(5%)
  colframe=blue!50}   % 테두리(기본 제목 배경) - 중간 파랑
% 회색 수식 박스 - graybox에 수식 정렬만 추가
\newtcolorbox{eqbox}{boxbase,
  ams align,          % 수식 정렬
  colback=gray!10,    % 본문 배경 - 회색(10%)
  colframe=gray!50}   % 테두리(기본 제목 배경) - 회색(50%)
 
%%%%%%%%%%%%%%%%%%%%%%%%%%%%%%%%%%%%%%%% 보충설명용 박스 제작
%% sssbox 정의
\NewTColorBox{sssbox}{g}{boxbase, % 제목을 {}안에 넣으면 제목이 함께 출력됨
  title={\boxtitle{\IfNoValueTF{#1}{Sss...}{Sss... \,\, #1}}},
  colframe=black,     % 테두리(기본 제목 배경) - 검정
  colbacktitle=white, % 제목 배경 변경 - 흰색
  coltitle=black}     % 제목 글씨색 - 검정

% {g}는 입력하지 않을 경우 아예 없는 것으로 판단함. 즉 괄호 자체가 없음
% {O{} g}는 입력하지 않으면 공백을 입력한 것으로 판단함. 즉 빈괄호{}를 생성함


%%%%%%%%%%%%%%%%%%%%%%%%%%%%%%%%%%%%%%%% 문제풀이용 박스 제작
%% checkbox 정의
% 번호 카운터 먼저 정의 - section 기준으로 리셋
% 번호 형식은 '챕터.섹션.누적번호'임 (예: 18.1.1)
\newcounter{checkbox}[section]
\renewcommand{\thecheckbox}{\arabic{chapter}.\arabic{section}.\arabic{checkbox}}
\NewTColorBox{checkboxbox}{g}{boxbase, % 제목을 {}안에 넣으면 제목이 함께 출력됨, 공통 서식 호출
  title={\boxtitle{check\thecheckbox\IfNoValueF{#1}{. \,\, #1}}},
  colframe=gray!50,   % 테두리(기본 제목 배경) - 회색(50%)
  coltitle=black}     % 제목 글씨색 - 검정
\newenvironment{checkbox}{\refstepcounter{checkbox}\begin{checkboxbox}}{\end{checkboxbox}}

%% checkbox* 정의 - 번호 없음
\NewTColorBox{checkboxbox*}{g}{boxbase,
  title={\boxtitle{check\IfNoValueF{#1}{. \,\, #1}}},
  colframe=gray!50,   % 테두리(기본 제목 배경) - 회색(50%)
  coltitle=black}     % 제목 글씨색 - 검정
\newenvironment{checkbox*}{\begin{checkboxbox*}}{\end{checkboxbox*}}

%% practicebox 정의
\newcounter{practicebox}[section]
\renewcommand{\thepracticebox}{\arabic{chapter}.\arabic{section}.\arabic{practicebox}}
\NewTColorBox{practiceboxbox}{g}{boxbase,
  title={\boxtitle{practice\thepracticebox\IfNoValueF{#1}{. \,\, #1}}},
  colframe=black,     % 테두리(기본 제목 배경) - 검정
  coltitle=white}     % 제목 글씨색 - 흰색
\newenvironment{practicebox}{\refstepcounter{practicebox}\begin{practiceboxbox}}{\end{practiceboxbox}}

%% practicebox* 정의 - 번호 없음
\NewTColorBox{practiceboxbox*}{g}{boxbase,
  title={\boxtitle{practice\IfNoValueF{#1}{. \,\, #1}}},
  colframe=black,     % 테두리(기본 제목 배경) - 검정
  coltitle=white}     % 제목 글씨색 - 흰색
\newenvironment{practicebox*}{\begin{practiceboxbox*}}{\end{practiceboxbox*}}

%% questionbox 정의
\NewTColorBox{questionboxbox}{g}{boxbase,
  title={\boxtitle{question\IfNoValueF{#1}{. \,\, #1}}},
  colframe=gray!50,   % 테두리(기본 제목 배경) - 회색(50%)
  coltitle=black}     % 제목 글씨색 - 검정
\newenvironment{questionbox}{\begin{questionboxbox}}{\end{questionboxbox}}
% 문제번호 저장용 매크로
\newcommand{\QuestionNumberumber}{0} % 초기값 0으로 설정
\newcommand{\QuestionNumber}[1]{\renewcommand{\QuestionNumberumber}{\unskip#1\unskip}} % 공백 무시, pn값을 저장

%% problembox 정의
\NewTColorBox{problemboxbox}{g}{boxbase,
  title={\boxtitle{problem\IfNoValueF{#1}{. \,\, #1}}},
  colframe=black,     % 테두리(기본 제목 배경) - 검정
  coltitle=white}     % 제목 글씨색 - 흰색
\newenvironment{problembox}{\begin{problemboxbox}}{\end{problemboxbox}}
% 문제번호 저장용 매크로
\newcommand{\ProblemNumberumber}{0} % 초기값 0으로 설정
\newcommand{\ProblemNumber}[1]{\renewcommand{\ProblemNumberumber}{\unskip#1\unskip}} % 공백 무시, pn값을 저장

%%%%%%%%%%%%%%%%%%%%%%%%%%%%%%%%%%%%%%%% seolmode와 호환되는 박스
%% solbox 정의 - 솔루션
% seolmode가 3,4일때만 출력하고 1,2일때는 사라짐
% \ifnum\seolmode>2 % seolmode가 3이상이면 출력
\NewTColorBox{solsolbox}{g}{boxbase,
  title={\boxtitle{Solution\IfNoValueF{#1}{. \,\, #1}}},
  colframe=blue!50,     % 중간 파란 테두리
  coltitle=black,       % 제목 글씨색
  colbacktitle=white}   % 제목 배경색
% 모드 변경용 하얀박스 - 위의 solsolbox와 완전 똑같이 만들고 전부 흰색으로 처리함
\NewTColorBox{whitesolsolbox}{g}{boxbase, 
  title={\boxtitle{\textcolor{white}{Solution\IfNoValueF{#1}{. \,\, #1}}}},
  colframe=white, coltitle=white,colbacktitle=white}
% 환경정의
\NewDocumentEnvironment{solbox}{g}{
  \ifnum\seolmode>2
    \begin{solsolbox}{#1}
  \else\begin{whitesolsolbox}{#1}\color{white}\fi}{
  \ifnum\seolmode>2
    \end{solsolbox}
  \else\end{whitesolsolbox}\fi}

% 솔루션 하이라이트, solight - 모드가 3 이상일때만 orange 출력
\newcommand{\soli}[1]{%
  \ifnum\seolmode>2 \hlo{#1}
  \else \hlp{#1} \fi}

%% 표 스타일 정의용 매크로

%% 자주 쓰는 열 정의
\newcolumntype{C}[1]{>{\centering\arraybackslash}p{#1}}   % 가운데 정렬 고정폭 열
\newcolumntype{L}[1]{>{\raggedright\arraybackslash}p{#1}} % 왼쪽 정렬 고정폭 열
\newcolumntype{R}[1]{>{\raggedleft\arraybackslash}p{#1}}  % 오른쪽 정렬 고정폭 열

%% 기본 테이블 여백 조정
\setlength{\tabcolsep}{8pt}  % 셀 좌우 간격
\renewcommand{\arraystretch}{1.2}  % 행 간격

%% 강조 셀 스타일 - 기존에 정의한 색 재사용 (이미 \definecolor 되어 있어야 함)
\newcommand{\graycell}[1]{\cellcolor{mygray!20}#1}   % 회색 강조 셀
\newcommand{\redcell}[1]{\cellcolor{myred!20}#1}     % 빨간 강조 셀
\newcommand{\bluecell}[1]{\cellcolor{myblue!20}#1}   % 파란 강조 셀


%% 그림이나 표를 감싸서 좌측 또는 우측에 배열하는 함수
\newcommand{\wrapfigR}[1]{ % fig나 tab을 우측에 붙이기
  \sbox0{#1}\begin{wrapfigure}{r}{\wd0}%
    \zz \usebox0 \zz \end{wrapfigure}} % 위아래 한 줄 들어가서 \zz로 없애기
\newcommand{\wrapfigL}[1]{ % fig나 tab을 좌측에 붙이기
  \sbox0{#1}\begin{wrapfigure}{l}{\wd0}%
    \zz \usebox0 \zz \end{wrapfigure}}


%%%%% 표
%% 표 내에서 정렬이 편하도록 함수 명명
\newcommand{\leftcell}[1]{\raggedright #1\arraybackslash}
\newcommand{\centercell}[1]{\centering #1\arraybackslash}
\newcommand{\rightcell}[1]{\raggedleft #1\arraybackslash}

% \tab[h]{캡션}{라벨}{정렬}{내용} - 내용 빼고 모두 생략 가능
\newcommand{\tab}[5][H]{ % 기본 위치 here. 무조건 강하게 H
  \begin{table}[#1] % 표 float 환경 시작, 위에서 H로 지정함. 안써도 됨
    \vspace{-3pt} % 표 위로 살짝 올리기
    \centering % 표 가운데 정렬
    \ifthenelse{\isempty{#2}}{}{ % 캡션 내용 비어있으면 생략
      \caption{#2} % 캡션이 비어 있지 않으면 삽입. tabular 아래에 넣으면 캡션은 아래로 감
    }
    \ifthenelse{\isempty{#3}}{}{%
      \label{tab:#3} % 라벨이 비어 있지 않으면 tab: 접두사와 함께 삽입
    }
    {\fontsize{8}{10}\selectfont % 본문폰트 크기 10pt, 표 폰트 크기 8pt
    \begin{tabular}{#4} % 표 열 형식 지정
      \hline
      #5 \\ % 표 내용
      \hline
    \end{tabular}}
    \vspace{3pt}% 하단 살짝 여백 넣음
  \end{table}}

%% table 선언 쉽도록 함수 작성 - \DeclareTable{라벨}{명령이름}{캡션}{열형식}{내용} 순으로 입력하면 됨
\newcommand{\DeclareTable}[5]{
  \newcommand{#2}{
    \begin{table}[H]
      \vspace{-3pt}
      \centering
      \ifstrempty{#3}{}{%
        \caption{#3}
      }
      \ifstrempty{#1}{}{%
        \label{tab:#1}
      }
      {\fontsize{8}{10}\selectfont % 본문폰트 크기 10pt, 표 폰트 크기 8pt
      \begin{tabular}{#4}
        \hline
        #5 \\
        \hline
      \end{tabular}}
      \vspace{3pt}% 하단 살짝 여백 넣음
    \end{table}}}

%%%%% 그림
%% 본래 그림을 다음과 같이 선언해야 함
% \newcommand{\kelvintemperatures}{ - 명령 이름 작성
%     \begin{figure}[h] \centering \includegraphics[scale=0.6]
%     {../figures/kelvin_temperatures} - 파일 경로(파일명)
%     \caption{Some temperatures on the Kelvin scale} - 캡션(파일명과 동일)
%     \label{fig:kelvin_temperatures} \end{figure} } - 라벨

% 규칙
% 1. 그림을 최초 호출할 때는 100% 크기로 호출함.
% 2. 본문에서 최종 호출할 때 80% 기본 크기로 호출함. 본문 폰트는 10pt이고, 그림은 80%가 되도록 함.
% 즉, 기본 호출 배열(1단계)은 1배인데 수정 가능하다.
%     최종 호출 배열(2단계)은 안쓰면 0.8배(1단계에서 0.8배)인데 수정 가능하도록함
%     2단계에서 [1]을 넣으면 그림이 커진다. 1단계에서 정한 100% 크기가 되는 것이다.

% 그림 계열
% 1. 일러스트레이터로 그린 경우 10pt로 작성하므로, 최초 호출 1배율([] 필요 없음), 본문 호출 0.8배율([] 필요 없음)
% 2. python으로 그린 경우도 10pt로 작성하므로, 최초 호출 1배율([] 필요 없음), 본문 호출 0.8배율([] 필요 없음)
% 3. 일반물리학(12판, fundamental)의 경우 크롬 pdf에서 150%로 캡쳐하면,
%    최초 호출 0.8배율일 때 본문과 폰트 크기 같아짐. 이후 본문호출 0.8배율
%    즉, 1단계에서 [0.8]로 크기 조정하고, 본문에서는 ([] 필요 없음)

\renewcommand{\figurename}{Fig.} % 전역적으로 그림을 Figure가 아닌 Fig.로 표시

%% figure 선언 쉽도록 함수 작성
% \DeclareFigure[<1단계 기본배율(기본 1)>]{파일명(라벨)}{명령이름}{캡션} 순으로 입력하면 됨
% 그림 크기가 최초에 보정이 필요한 경우 \DeclareFigure[0.8]{}{\aaa}{}와 같이 시작함.
% 최초 배율(1단계)이 100%이 되도록 할 것
% 값을 쓰지 않는 경우 1배율([1])로 들어감.
% 이후 본문에서 \aaa[0.8] 으로 최종 배율(2단계) 호출
\NewDocumentCommand{\DeclareFigure}{O{1} m m m}{%
  \NewDocumentCommand{#3}{O{0.8}}{%
    \begin{figure}[H]\centering
      \includegraphics[scale=\fpeval{#1*(##1)}]{figures/#2}%
      \caption{#4}\label{fig:#2}%
    \end{figure}}}

%% figure - 번호, 라벨 없는 Notag 버전 - section 번호 누적되지 않도록 하기 위함
\NewDocumentCommand{\DeclareFigureNoTag}{O{1} m m m}{%
  \NewDocumentCommand{#3}{O{0.8}}{%
    \begin{figure}[H]\centering
      \includegraphics[scale=\fpeval{#1*(##1)}]{figures/#2}%
      \caption*{#4}% 번호 없음
    \end{figure}}}

%% 라벨 없이 간단하게 삽입하는 경우 - notag보다 유용
% \quickfig{파일명}으로 부를 수 있음.
% 기본 0.8배 크기로 삽입되지만 \quickfig[0.6]{파일명} 이런식으로 사이즈 변경 가능. [1]넣으면 크기 커짐(100%)
\NewDocumentCommand{\quickfig}{O{0.8} m}{%
  \begin{center}
    \includegraphics[scale=#1]{figures/#2}
  \end{center}}

% memo와 비슷하게 홀수짝수일 때 페이지 측면으로 붙는 그림 - 넘어가는 경우 shift되게 하여 항상 페이지 안쪽 유지
% memo와 위치가 겹치므로, 항상 background로 보내야 함. 메모가 위로 오는 게 더 중요함.
\newcounter{edgefiguremark}
\NewDocumentCommand{\DeclareEdgeFigure}{O{1} m m m}{%
  \NewDocumentCommand{#3}{O{0.8}}{%
    \stepcounter{edgefiguremark}%
    \tikz[remember picture,baseline]{\coordinate (edgefiguremark-\theedgefiguremark) at (0,0);}%
\MeasureHalfFrom{ % 실제 박스와 똑같이 측정
  \begin{minipage}{130pt}\centering
    \includegraphics[scale=\fpeval{#1*(##1)}]{figures/#2}\\[2pt]
    {\captionsetup{type=figure,width=\linewidth,hypcap=false}%
     \caption*{#4}} % 번호 누적 없이 캡션 효과 고려. 라벨 없음
  \end{minipage}}%
    \CoordsHere % 좌표 계산
    \ClampFromCoords % 보정량 산출(오프셋 반영)
%    \AddToShipoutPictureBG*{ % 가장 아래층으로 내림 - 그림 전체가 이동하는 문제 발생
  \begin{tikzpicture}[remember picture,overlay]%
    \checkoddpage\ifoddpage % 홀수 페이지 - 오른쪽 엣지로 이동
      \node[yshift=\yshiftEdge] at ($(current page.east |- edgefiguremark-\theedgefiguremark)+(-82pt,0)$) {%
        \begin{minipage}{130pt}\centering
          \includegraphics[scale=\fpeval{#1*(##1)}]{figures/#2}\\[2pt]
          \phantomsection
          {\captionsetup{type=figure,width=\linewidth,hypcap=false}%
           \phantomsection\captionof{figure}{#4}\label{fig:#2}}%
        \end{minipage}};
    \else % 짝수 페이지 - 왼쪽 엣지로 이동
      \node[yshift=\yshiftEdge] at ($(current page.west |- edgefiguremark-\theedgefiguremark)+(82pt,0)$) {%
        \begin{minipage}{130pt}\centering
          \includegraphics[scale=\fpeval{#1*(##1)}]{figures/#2}\\[2pt]
          \phantomsection
          {\captionsetup{type=figure,width=\linewidth,hypcap=false}%
           \phantomsection\captionof{figure}{#4}\label{fig:#2}}%
        \end{minipage}};
    \fi
  \end{tikzpicture}%
%} % 가장 아래층 내림 안됨
\ignorespaces}}

% 교사용 seolfig - 모드 3,4일때만 나오는 그림. 1,2에서는 자리만 차지
\NewDocumentCommand{\seolfig}{O{0.8} m}{%
  \begin{center}
    \ifnum\seolmode>2
      \includegraphics[scale=#1]{figures/#2}%
    \else
      \phantom{\includegraphics[scale=#1]{figures/#2}}%
    \fi \end{center}}

%% figure 2개짜리 함수 작성(파일명은 같고_a, _b만 달라야 함)
% \DeclareDoubleFigure[<1단계 기본배율(기본 1)>]{파일명(라벨)}{명령이름}{캡션} 순으로 입력하면 됨
\NewDocumentCommand{\DeclareDoubleFigure}{O{1} m m m}{%
  \NewDocumentCommand{#3}{O{0.8}}{%
    \begin{figure}[H] \centering
      \begin{subfigure}[b]{0.3\textwidth} % bottm으로 정렬(t,c 가능), 폭 지정 0.3
        \includegraphics[scale=\fpeval{#1*(##1)}]{figures/#2_a}%
        \caption{}\label{fig:#2:a} % a, b, c 순차적으로 캡션 붙음, 하나씩 라벨 호출할 수 있음
      \end{subfigure}
      \begin{subfigure}[b]{0.3\textwidth}
        \includegraphics[scale=\fpeval{#1*(##1)}]{figures/#2_b}%
        \caption{}\label{fig:#2:b}
      \end{subfigure}
      \caption{#4}
      \label{fig:#2}
    \end{figure}}}

%% figure 3개짜리 함수 작성(파일명은 같고_a, _b, _c만 달라야 함) - \DeclareTripleFigure{파일명(라벨)}{명령이름}{캡션} 순으로 입력하면 됨
\NewDocumentCommand{\DeclareTripleFigure}{O{1} m m m}{%
  \NewDocumentCommand{#3}{O{0.8}}{%
    \begin{figure}[H] \centering
      \begin{subfigure}[b]{0.3\textwidth} % bottm으로 정렬(t,c 가능), 폭 지정 0.3
        \includegraphics[scale=\fpeval{#1*(##1)}]{figures/#2_a}%
        \caption{}\label{fig:#2:a} % a, b, c 순차적으로 캡션 붙음, 하나씩 라벨 호출할 수 있음
      \end{subfigure}
      \begin{subfigure}[b]{0.3\textwidth}
        \includegraphics[scale=\fpeval{#1*(##1)}]{figures/#2_b}%
        \caption{}\label{fig:#2:b}
      \end{subfigure}
      \begin{subfigure}[b]{0.3\textwidth}
        \includegraphics[scale=\fpeval{#1*(##1)}]{figures/#2_c}%
        \caption{}\label{fig:#2:c}
      \end{subfigure}
      \caption{#4}
      \label{fig:#2}
    \end{figure}}}


%% figure 6개짜리 함수 작성 (파일명은 같고 _a ~ _f까지 순차적으로, 3개씩 2줄 배치)
%% 사용법: \DeclareQuadFigure{파일명(라벨)}{명령이름}{캡션}
\NewDocumentCommand{\DeclareQuadFigure}{O{1} m m m}{%
  \NewDocumentCommand{#3}{O{0.8}}{%
    \begin{figure}[H] \centering
      % 1행: a, b, c
      \begin{subfigure}[b]{0.3\textwidth}  % 첫 번째 줄: a, b, c
        \includegraphics[scale=\fpeval{#1*(##1)}]{figures/#2_a}%
        \caption{}\label{fig:#2:a}
      \end{subfigure}
      \begin{subfigure}[b]{0.3\textwidth}
        \includegraphics[scale=\fpeval{#1*(##1)}]{figures/#2_b}%
        \caption{}\label{fig:#2:b}
      \end{subfigure}
      \begin{subfigure}[b]{0.3\textwidth}
        \includegraphics[scale=\fpeval{#1*(##1)}]{figures/#2_c}%
        \caption{}\label{fig:#2:c}
      \end{subfigure}
      \par\medskip % 줄바꿈 + 여백
      \begin{subfigure}[b]{0.3\textwidth}  % 두 번째 줄: d, e, f
        \includegraphics[scale=\fpeval{#1*(##1)}]{figures/#2_d}%
        \caption{}\label{fig:#2:d}
      \end{subfigure}
      \begin{subfigure}[b]{0.3\textwidth}
        \includegraphics[scale=\fpeval{#1*(##1)}]{figures/#2_e}%
        \caption{}\label{fig:#2:e}
      \end{subfigure}
      \begin{subfigure}[b]{0.3\textwidth}
        \includegraphics[scale=\fpeval{#1*(##1)}]{figures/#2_f}%
        \caption{}\label{fig:#2:f}
      \end{subfigure}
      \caption{#4}
      \label{fig:#2}
    \end{figure}}}
%% \seolfa{단어} 형태로 사용
%% 출력 모드는 \seolmode 값으로 제어
% 1=그대로 출력
% 2=( 글자 숨김 ), 학생용 빈칸
% 3=( 붉게 강조 ), 답안배포용
% 4=( 붉게 강조 ) 및 메모 노출, 교사용

\newcount\seolmode

\newcommand{\seolfa}[1]{% a는 괄호가 있음. mbox로 하나의 덩어리로 묶어서 줄바꿈 되지 않게 함.
  \ifnum\seolmode=1
    \mbox{\hly{\bd{#1}}}%
  \else\ifnum\seolmode=2
    \mbox{\bd{(\quad \phantom{#1} \quad)}}%
  \else\ifnum\seolmode=3
    \mbox{\textcolor{red}{\bd{(\quad \hly{#1} \quad)}}}%
  \else\ifnum\seolmode=4
    \mbox{\textcolor{red}{\bd{(\quad \hly{#1} \quad)}}}%
  \fi\fi\fi\fi
}

\newcommand{\seolfb}[1]{% b는 괄호가 없음. 표 안의 내용 등에 활용. 줄바꿈 가능
  \ifnum\seolmode=1
    \hly{\bd{#1}}%
  \else\ifnum\seolmode=2
    \hlw{\textcolor{white}{\bd{#1}}}%
  \else\ifnum\seolmode=3
    \hly{\textcolor{red}{\bd{#1}}}%
  \else\ifnum\seolmode=4
    \hly{\textcolor{red}{\bd{#1}}}%
  \fi\fi\fi\fi
}

%% 교사용 녹색 하이라이트 (seolmode가 4일 때는 \hll(라임형광펜), 나머지는 \hlw(화이트) - 똑같이 여백 차지하게 하려고 hlw 씀
\newcommand{\seoli}[1]{%
  \ifnum\seolmode=4 \hll{#1}%
  \else \hlw{#1}\fi}

%% memo 마커 (seolmode가 4일 때만 출력하고 1,2,3일때는 사라짐 - 처음부터 공간 차지 안함)
\newcommand{\memo}[1]{%
  \ifnum\seolmode=4 \zz % 항상 zz로 한 줄 없애고 시작하기. 언제나 독립적으로 위아래가 비워진 상태로 memo 삽입해야 함
    \tikz[overlay]{% 현재 줄 기준 그림을 겹쳐서 출력 (본문 레이아웃에 영향 없음)
      \ifodd\value{page} % 페이지 번호가 홀수이면
        \node[anchor=base west, xshift=+382pt, text width=140pt, align=left] at (0,0) {%
          \scriptsize\fboxsep=0pt % 작은 글자로 크기 설정, 박스 여백 0으로 설정
          \colorbox{yellow!20}{\parbox{140pt}{#1}}}; % 노란색 배경 박스에 입력된 내용 출력
      \else % 페이지 번호가 짝수이면
        \node[anchor=base west, xshift=-159pt, text width=140pt, align=left] at (0,0) {%
          \scriptsize\fboxsep=0pt
          \colorbox{yellow!20}{\parbox{140pt}{#1}}};\fi}\fi\ignorespaces}%

% 솔루션 하이라이트, solight - 모드가 3 이상일때만 orange 출력
\newcommand{\soli}[1]{%
  \ifnum\seolmode>2
    \highLight[{[RGB]{255,220,70}}]{#1}%
  \else
    #1%
  \fi}

% 답안용 파란 테두리 솔루션박스% 답안용 파란 테두리 솔루션박스 (seolmode가 3,4일때만 출력하고 1,2일때는 사라짐)
% \ifnum\seolmode>2 % seolmode가 3이상이면 출력
\NewTColorBox{solsolbox}{O{} g}{ % 제목을 {}안에 넣으면 제목이 함께 출력됨
  title={\raisebox{0pt}[9pt][2pt]{Solution \IfNoValueF{#2}{ \,-\, #2}}},
  colback=white,        % 흰색 배경
  colframe=blue!50,     % 중간 파란 테두리
  coltitle=black,       % 제목 글씨색
  colbacktitle=white,   % 제목 배경색
  boxrule=1pt,          % 굵은 테두리 선
  arc=2mm,              % 살짝 둥근 모서리
  left=4pt, right=4pt,  % 좌우 여백
  top=4pt, bottom=4pt,   % 상하 여백
  before upper={ % 박스 안에서도 위아래 여백이 본문과 동일하게 세팅되어야 함.
  \setlength{\lineskiplimit}{2pt} % 인라인 위아래 간격이 임계값보다 작으면
  \setlength{\lineskip}{5pt} % 추가 간격 확보(수식 입력 등에 유용함)
  \setlength{\abovedisplayskip}{5pt} % 일반적인 디스플레이 수식 여백 설정
  \setlength{\belowdisplayskip}{5pt}
  \setlength{\abovedisplayshortskip}{5pt} % 특수상황(수식 앞줄이 짧게 끝나는 경우)
  \setlength{\belowdisplayshortskip}{5pt}}} % 특수상황(수식 아랫줄이 짧게 끝나는 경우)
% 모드 변경용 하얀박스
\NewTColorBox{whitesolsolbox}{O{} g}{ % 위의 solsolbox와 완전 똑같이 만들고 전부 흰색으로 처리함
  title={\raisebox{0pt}[9pt][2pt]{\textcolor{white}{Solution \IfNoValueF{#2}{ \,-\, #2}}}}, 
  colback=white, colframe=white, coltitle=white, colbacktitle=white,boxrule=1pt, arc=2mm, left=4pt, right=4pt, top=4pt, bottom=4pt,before upper={\setlength{\lineskiplimit}{2pt}\setlength{\lineskip}{5pt}\setlength{\abovedisplayskip}{5pt}\setlength{\belowdisplayskip}{5pt}\setlength{\abovedisplayshortskip}{5pt}\setlength{\belowdisplayshortskip}{5pt}}}


% 환경정의
\NewDocumentEnvironment{solbox}{O{} g}{%
  \ifnum\seolmode>2
    \begin{solsolbox}[#1]{#2}%
  \else\begin{whitesolsolbox}[#1]{#2}\color{white}\fi}{%
  \ifnum\seolmode>2
    \end{solsolbox}%
  \else\end{whitesolsolbox}\fi}


% 교사용 페이지 삽입 명령 - 3 이상일 때 노출됨
\newcommand{\insertTeacherPages}[2]{%
  \ifnum\seolmode>2
    \clearpage
    \thispagestyle{empty}
    {\centering\red{\Large\bd{[ 교사용 페이지(양면) 1/2 ]}} \par}
    \vspace{1em}
    \normalsize #1
    \clearpage
    \thispagestyle{empty}
    {\centering\red{\Large\bd{[ 교사용 페이지(양면) 2/2 ]}} \par}
    \vspace{1em}
    \normalsize #2
    \clearpage
    \addtocounter{page}{-2}
  \fi
}
 % ddd

%%%%%%%%%%%%%%%%%%%%%%%%%%%%%%%%%%%%%%%% 페이지 설정
\usepackage[a4paper, % 기본용지 설정
top=25mm, bottom=25mm, left=20mm, right=20mm]{geometry}

%%%%%%%%%%%%%%%%%%%%%%%%%%%%%%%%%%%%%%%% 경고 무시
\hbadness=10000 % underfull hbox 경고 무시(수평 방향 여백 경고 badness 10000까지 무시)
\vbadness=10000 % overfull hbox 경고 무시(수직 방향 페이니 나눔 단락 경고 badness 10000까지 무시)
\hfuzz=7pt  % 7pt 이하의 overfull \hbox 경고 무시

% 제목, 서문, 목차 불러오는 함수 만들기
\newcommand{\includeHead}[1]{\input{../includes/#1.tex}}

% 참고문헌
\newcommand{\setupBib}{ % 참고문헌 있는 경우 불러오기
  \usepackage[backend=biber, style=numeric, sorting=none]{biblatex}
  \usepackage{csquotes}
  \addbibresource{../bib/papers.bib}
  \addbibresource{../bib/references.bib}
  \addbibresource{../bib/websites.bib}}
\newcommand{\printBib}{ % sloppy로 참고문헌 출력시 길이가 길다면 강제 줄바꿈
  \begin{sloppypar} \printbibliography \end{sloppypar}}

% 챕터 불러오기
\newcommand{\loadchapter}[1]{\subimport{../chapters/#1/}{chapter.tex}}
% 히든 챕터 불러오기 - 안쓰는 챕터지만 식 참조 위하여 외부참조(aux 파일 한 번 컴파일 해야 함)
\newcommand{\hiddenchapter}[1]{%
  \externaldocument[]{../chapters/#1/chapter}%
}

%%%%%%%%%%%%%%%%%%%%%%%%%%%%%%%%%%%%%%%% 캡션, 링크
\renewcommand{\equationautorefname}{Eq.} % 수식 참조 시 equation이 아닌 Eq. 로 나옴. aliascnt 필요
\renewcommand{\figureautorefname}{Fig.} % figure는 Fig
\renewcommand{\tableautorefname}{Table.} % table은 Table 그대로
\captionsetup[figure]{name=Fig.} % 그림 아래 캡션도 Fig. 으로 나오도록 함. caption 필요
\captionsetup[figure]{width=0.8\linewidth} % 캡션 너비 조절
\captionsetup[table]{width=0.8\linewidth}
\numberwithin{equation}{section} % eq, figure, table이 표시되는 것을 section까지 설정함. 설정하지 않으면 18.1처럼 chapter 별로 정리됨
\numberwithin{figure}{section}
\numberwithin{table}{section}
\setlength{\abovecaptionskip}{3pt}    % 캡션 위 간격
\setlength{\belowcaptionskip}{-12pt}  % 캡션 아래 간격
